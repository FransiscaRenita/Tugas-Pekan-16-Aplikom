\documentclass[a4paper,10pt]{article}
\usepackage{eumat}

\begin{document}
\begin{eulernotebook}
\begin{eulercomment}
Nama : Fransisca Renita Pejoresa\\
Kelas: Matematika E 2022\\
NIM  : 22305144012

\begin{eulercomment}
\eulerheading{Menggambar Plot 3D dengan EMT}
\begin{eulercomment}
\end{eulercomment}
\eulersubheading{1. Menggambar Grafik Fungsi Dua Variabel * dalam Bentuk Ekspresi}
\begin{eulercomment}
Langsung

Fungsi Dua Variabel didefinisikan sebagai sebuah fungsi bernilai real
dari dua variabel real, yakni fungsi f yang memadankan setiap pasangan\\
terurut (x,y) pada suatu himpunan D dari bidang dengan bilangan real\\
tunggal f (x,y).

Di dalam program numerik EMT, ekspresi adalah string. Jika ditandai
sebagai simbolis, mereka akan mencetak melalui Maxima, jika tidak
melalui EMT. Ekspresi dalam string digunakan untuk membuat plot dan
banyak fungsi numerik. Untuk ini, variabel dalam ekspresi harus "x"
dan "y".

Untuk grafik suatu fungsi, gunakan

- ekspresi sederhana dalam x dan y,\\
- nama fungsi dari dua variabel\\
- atau matriks data.

contoh:
\end{eulercomment}
\begin{eulerprompt}
>plot3d("x^2+y^2"):
\end{eulerprompt}
\eulerimg{27}{images/Fransisca Renita_22305144012_Tugas 2-001.png}
\begin{eulerprompt}
>plot3d("x^2+5*y^2"):
\end{eulerprompt}
\eulerimg{27}{images/Fransisca Renita_22305144012_Tugas 2-002.png}
\begin{eulerprompt}
>plot3d("x^2*y+3*y^2"):
\end{eulerprompt}
\eulerimg{27}{images/Fransisca Renita_22305144012_Tugas 2-003.png}
\begin{eulerprompt}
>aspect(1.5); plot3d("x^2+sin(y)",-5,5,0,6*pi):
\end{eulerprompt}
\eulerimg{17}{images/Fransisca Renita_22305144012_Tugas 2-004.png}
\begin{eulercomment}
1. aspect(1.5) mengatur aspek rasio pada grafik 3D.\\
2. plot3d("x\textasciicircum{}2+sin(y)",-5,5,0,6*pi) adalah fungsi matematika yang
digunakan untuk membuat grafik 3D.\\
3. -5,5 mengatur rentang sumbu x yang akan ditampilkan pada grafik.\\
4. 0,6*pi mengatur rentang sumbu y yang akan ditampilkan pada grafik.
\end{eulercomment}
\begin{eulercomment}

\end{eulercomment}
\eulersubheading{Fungsi umum untuk plot 3D.}
\begin{eulercomment}
Fungsi plot3d (x, y, z, xmin, xmax, ymin, ymax, n, a,  ..\\
b, c, d, r, scale, fscale, frame, angle, height, zoom, distance, ..)

Rentang plot untuk fungsi dapat ditentukan dengan\\
- a,b: rentang x\\
- c,d: rentang y\\
- r : persegi simetris di sekitar (0,0).\\
- n : jumlah subinterval untuk plot.

Ada beberapa parameter untuk menskalakan fungsi atau mengubah tampilan
grafik.\\
- fscale: menskalakan ke nilai fungsi (defaultnya adalah \textless{}fscale).\\
- scale: angka atau vektor 1x2 untuk menskalakan ke arah x dan y.\\
- frame: jenis bingkai (default 1).

Tampilan dapat diubah dengan berbagai cara.\\
- distance: jarak pandang ke plot.\\
- zoom: nilai zoom.\\
- angle: sudut terhadap sumbu y negatif dalam radian.\\
- height: ketinggian pandangan dalam radian.

Nilai default dapat diperiksa atau diubah dengan fungsi view(). Ini
mengembalikan parameter dalam urutan di atas.

\end{eulercomment}
\begin{eulerprompt}
>view 
\end{eulerprompt}
\begin{euleroutput}
  [5,  2.6,  2,  0.4]
\end{euleroutput}
\begin{eulercomment}
Jarak yang lebih dekat membutuhkan lebih sedikit zoom. Efeknya lebih
seperti lensa sudut lebar.

contoh:
\end{eulercomment}
\begin{eulerprompt}
>plot3d("exp(-(x^2+y^2)/5)",r=10,n=80,fscale=4,scale=1.2,frame=3,>user):
\end{eulerprompt}
\eulerimg{17}{images/Fransisca Renita_22305144012_Tugas 2-005.png}
\begin{eulercomment}
1. exp(-(x\textasciicircum{}2+y\textasciicircum{}2)/5) adalah fungsi matematika yang digunakan untuk
membuat grafik 3D.\\
2. r=10 mengatur jarak maksimum dari pusat grafik ke tepi grafik.\\
3. n=80 mengatur jumlah titik yang digunakan untuk membuat grafik.\\
4. fscale=4 mengatur faktor skala untuk warna.\\
5. scale=1.2 mengatur faktor skala untuk ukuran grafik.\\
6. frame=3 mengatur jenis bingkai yang digunakan untuk grafik.

\end{eulercomment}
\begin{eulercomment}

Pada contoh berikut, sudut=0 dan tinggi=0 dilihat dari sumbu y
negatif. Label sumbu untuk y disembunyikan dalam kasus ini.
\end{eulercomment}
\begin{eulerprompt}
>plot3d("x^2+y",distance=3,zoom=1,angle=pi/2,height=0):
\end{eulerprompt}
\eulerimg{17}{images/Fransisca Renita_22305144012_Tugas 2-006.png}
\begin{eulercomment}
1. x\textasciicircum{}2+y adalah fungsi matematika yang digunakan untuk membuat grafik
3D.\\
2. distance=3 mengatur jarak pandang dari grafik.\\
3. zoom=1 mengatur faktor perbesaran grafik.\\
4. angle=pi/2 mengatur sudut pandang grafik dalam radian.\\
5. height=0 mengatur ketinggian pandangan dari grafik.

Plot selalu terlihat berada di tengah kubus plot. Anda dapat
memindahkan bagian tengah dengan parameter tengah.
\end{eulercomment}
\begin{eulerprompt}
>plot3d("x^4+y^2",a=0,b=1,c=-1,d=1,angle=-20°,height=20°, ...
>  center=[0.4,0,0],zoom=5):
\end{eulerprompt}
\eulerimg{17}{images/Fransisca Renita_22305144012_Tugas 2-007.png}
\begin{eulercomment}
1. x\textasciicircum{}4+y\textasciicircum{}2 adalah fungsi matematika yang digunakan untuk membuat
grafik 3D.\\
2. a=0,b=1,c=-1,d=1 mengatur rentang sumbu x dan y yang akan
ditampilkan pada grafik.\\
3. angle=-20° mengatur sudut pandang grafik dalam derajat.\\
4. height=20° mengatur ketinggian pandangan dari grafik dalam derajat.\\
5. center=[0.4,0,0] mengatur pusat pandangan dari grafik.\\
6. zoom=5 mengatur faktor perbesaran grafik.

Plotnya diskalakan agar sesuai dengan unit kubus untuk dilihat. Jadi
tidak perlu mengubah jarak atau zoom tergantung ukuran plot. Namun
labelnya mengacu pada ukuran sebenarnya.

Jika Anda mematikannya dengan scale=false, Anda harus berhati-hati
agar plot tetap masuk ke dalam jendela plotting, dengan mengubah jarak
pandang atau zoom, dan memindahkan bagian tengah.

\end{eulercomment}
\begin{eulerprompt}
>plot3d("5*exp(-x^2-y^2)",r=2,<fscale,<scale,distance=13,height=50°, ...
>  center=[0,0,-2],frame=3):
\end{eulerprompt}
\eulerimg{17}{images/Fransisca Renita_22305144012_Tugas 2-008.png}
\begin{eulercomment}
1. 5*exp(-x\textasciicircum{}2-y\textasciicircum{}2) adalah fungsi matematika yang digunakan untuk
membuat grafik 3D.\\
2. r=2 mengatur jarak maksimum dari pusat grafik ke tepi grafik.\\
3. \textless{}fscale mengatur faktor skala untuk warna.\\
4. \textless{}scale mengatur faktor skala untuk ukuran grafik.\\
5. distance=13 mengatur jarak pandang dari grafik.\\
6. height=50° mengatur ketinggian pandangan dari grafik dalam derajat.\\
7. center=[0,0,-2] mengatur pusat pandangan dari grafik.\\
8. frame=3 mengatur jenis bingkai yang digunakan untuk grafik.

Plot kutub juga tersedia. Parameter polar=true menggambar plot kutub.
Fungsi tersebut harus tetap merupakan fungsi dari x dan y. Parameter
"fscale" menskalakan fungsi dengan skalanya sendiri. Kalau tidak,
fungsinya akan diskalakan agar sesuai dengan kubus.
\end{eulercomment}
\begin{eulerprompt}
>plot3d("1/(x^2+y^2+1)",r=5,>polar, ...
>fscale=2,>hue,n=100,zoom=4,>contour,color=blue):
\end{eulerprompt}
\eulerimg{17}{images/Fransisca Renita_22305144012_Tugas 2-009.png}
\begin{eulercomment}
1. 1/(x\textasciicircum{}2+y\textasciicircum{}2+1) adalah fungsi matematika yang digunakan untuk membuat
grafik 3D.\\
2. r=5 mengatur jarak maksimum dari pusat grafik ke tepi grafik.\\
3. polar mengatur tampilan grafik dalam koordinat polar.\\
4. fscale=2 mengatur faktor skala untuk warna.\\
5. hue mengatur skala warna yang digunakan pada grafik.\\
6. n=100 mengatur jumlah titik yang digunakan untuk membuat grafik.\\
7. zoom=4 mengatur faktor perbesaran grafik.\\
8. contour mengatur tampilan garis kontur pada grafik.\\
9. color=blue mengatur warna garis kontur pada grafik.
\end{eulercomment}
\begin{eulerprompt}
>function f(r) := exp(-r/2)*cos(r); ...
>plot3d"f(x^2+y^2)",>polar,scale=[1,1,0.4],r=pi,frame=3,zoom=4):
\end{eulerprompt}
\begin{euleroutput}
  Function plot3d needs at least one argument!
  Use: plot3d (x \{, y: none, z: none, xmin: none, xmax: none, ...\}) 
  Error in:
  plot3d"f(x^2+y^2)",>polar,scale=[1,1,0.4],r=pi,frame=3,zoom=4) ...
        ^
\end{euleroutput}
\begin{eulerprompt}
>function f(r) := exp(-r/2)*cos(r); ...
>plot3d("f(x^2+y^2)",>polar,scale=[1,1,0.4],r=pi,frame=3,zoom=4):
\end{eulerprompt}
\eulerimg{17}{images/Fransisca Renita_22305144012_Tugas 2-010.png}
\begin{eulercomment}
1. function f(r) := exp(-r/2)*cos(r) adalah fungsi matematika yang
didefinisikan sebagai f(r) = e\textasciicircum{}(-r/2) * cos(r).\\
2. plot3d("f(x\textasciicircum{}2+y\textasciicircum{}2)",polar,scale=[1,1,0.4],r=pi,frame=3,zoom=4)
adalah perintah untuk membuat grafik 3D dari fungsi f(x\textasciicircum{}2+y\textasciicircum{}2).\\
3. polar mengatur tampilan grafik dalam koordinat polar.\\
4. scale=[1,1,0.4] mengatur faktor skala untuk ukuran grafik.\\
5. r=pi mengatur jarak maksimum dari pusat grafik ke tepi grafik.\\
6. frame=3 mengatur jenis bingkai yang digunakan untuk grafik.\\
7. zoom=4 mengatur faktor perbesaran grafik.

Parameter memutar memutar fungsi di x di sekitar sumbu x.

- rotate=1: Menggunakan sumbu x\\
- rotate=2: Menggunakan sumbu z
\end{eulercomment}
\begin{eulerprompt}
>plot3d("x^2+1",a=-1,b=1,rotate=true,grid=5):
\end{eulerprompt}
\eulerimg{17}{images/Fransisca Renita_22305144012_Tugas 2-011.png}
\begin{eulercomment}
1. x\textasciicircum{}2+1 adalah fungsi matematika yang digunakan untuk membuat grafik
3D.\\
2. a=-1,b=1 mengatur rentang sumbu x yang akan ditampilkan pada
grafik.\\
3. rotate=true mengatur grafik agar dapat diputar secara interaktif.\\
4. grid=5 mengatur jumlah garis koordinat yang ditampilkan pada
grafik.
\end{eulercomment}
\begin{eulerprompt}
>plot3d("x^2+1",a=-1,b=1,rotate=2,grid=5):
\end{eulerprompt}
\eulerimg{17}{images/Fransisca Renita_22305144012_Tugas 2-012.png}
\begin{eulercomment}
1. x\textasciicircum{}2+1 adalah fungsi matematika yang digunakan untuk membuat grafik
3D.\\
2. a=-1,b=1 mengatur rentang sumbu x yang akan ditampilkan pada
grafik.\\
3. rotate=2 mengatur grafik agar dapat diputar secara interaktif
dengan menggunakan mouse.\\
4. grid=5 mengatur jumlah garis koordinat yang ditampilkan pada
grafik.
\end{eulercomment}
\begin{eulerprompt}
>plot3d("sqrt(25-x^2)",a=0,b=5,rotate=1):
\end{eulerprompt}
\eulerimg{17}{images/Fransisca Renita_22305144012_Tugas 2-013.png}
\begin{eulercomment}
1. sqrt(25-x\textasciicircum{}2) adalah fungsi matematika yang digunakan untuk membuat
grafik 3D.\\
2. a=0,b=5 mengatur rentang sumbu x yang akan ditampilkan pada grafik.\\
3. rotate=1 mengatur grafik agar dapat diputar secara interaktif.
\end{eulercomment}
\begin{eulerprompt}
>plot3d("x*sin(x)",a=0,b=6pi,rotate=2):
\end{eulerprompt}
\eulerimg{17}{images/Fransisca Renita_22305144012_Tugas 2-014.png}
\begin{eulercomment}
1. x*sin(x) adalah fungsi matematika yang digunakan untuk membuat
grafik 3D.\\
2. a=0,b=6pi mengatur rentang sumbu x yang akan ditampilkan pada
grafik.\\
3. rotate=2 mengatur grafik agar dapat diputar secara interaktif.

Berikut adalah plot dengan tiga fungsi.
\end{eulercomment}
\begin{eulerprompt}
>plot3d("x","x^2+y^2","y",r=2,zoom=3.5,frame=3):
\end{eulerprompt}
\eulerimg{17}{images/Fransisca Renita_22305144012_Tugas 2-015.png}
\begin{eulercomment}
1. x adalah fungsi matematika yang digunakan untuk menentukan nilai
sumbu x pada grafik.\\
2. x\textasciicircum{}2+y\textasciicircum{}2 adalah fungsi matematika yang digunakan untuk menentukan
nilai sumbu z pada grafik.\\
3. y adalah fungsi matematika yang digunakan untuk menentukan nilai
sumbu y pada grafik.\\
4. r=2 mengatur jarak maksimum dari pusat grafik ke tepi grafik.\\
5. zoom=3.5 mengatur faktor perbesaran grafik.\\
6. frame=3 mengatur jenis bingkai yang digunakan untuk grafik.
\end{eulercomment}
\begin{eulercomment}


\begin{eulercomment}
\eulerheading{2. Menggambar Grafik Fungsi Dua Variabel yang}
\begin{eulercomment}
* Rumusnya Disimpan dalam Variabel Ekspresi

Fungsi ini dapat memplot plot 3D dengan grafik fungsi dua\\
variabel, permukaan berparameter, kurva ruang, awan titik,\\
penyelesaian persamaan tiga variabel. Semua plot 3D bisa\\
ditampilkan sebagai anaglyph.

fungsi plot3d (x, y, z, xmin, xmax, ymin, ymax, n, a

Parameter

x : ekspresi dalam x dan y\\
x,y,z : matriks koordinat suatu permukaan\\
x,y,z : ekspresi dalam x dan y untuk permukaan parametrik\\
x,y,z : ekspresi dalam x untuk memplot kurva ruang

xmin,xmax,ymin,ymax :\\
\end{eulercomment}
\begin{eulerttcomment}
  x,y batas ekspresi
\end{eulerttcomment}
\begin{eulercomment}

contoh:

ekspresi dalam string
\end{eulercomment}
\begin{eulerprompt}
>expr := "x^2+sin(y)"
\end{eulerprompt}
\begin{euleroutput}
  x^2+sin(y)
\end{euleroutput}
\begin{eulercomment}
plot ekspresi
\end{eulercomment}
\begin{eulerprompt}
>plot3d(expr,-5,5,0,6*pi):
\end{eulerprompt}
\eulerimg{17}{images/Fransisca Renita_22305144012_Tugas 2-016.png}
\begin{eulercomment}
1. x\textasciicircum{}2+sin(y) adalah fungsi matematika yang digunakan untuk membuat
grafik 3D.\\
2. -5,5 mengatur rentang sumbu x yang akan ditampilkan pada grafik.\\
3. 0,6*pi mengatur rentang sumbu y yang akan ditampilkan pada grafik.

contoh 1:
\end{eulercomment}
\begin{eulerprompt}
> 
>expr := "4*x^3*y"
\end{eulerprompt}
\begin{euleroutput}
  4*x^3*y
\end{euleroutput}
\begin{eulerprompt}
>aspect(1.5); plot3d(expr):
\end{eulerprompt}
\eulerimg{17}{images/Fransisca Renita_22305144012_Tugas 2-017.png}
\begin{eulercomment}
1. aspect(2) mengatur aspek rasio pada grafik 3D.\\
2. plot3d(expr) adalah fungsi matematika yang digunakan untuk membuat
grafik 3D.
\end{eulercomment}
\begin{eulercomment}
contoh 2:
\end{eulercomment}
\begin{eulerprompt}
>expr := "cos(x)*sin(y)"
\end{eulerprompt}
\begin{euleroutput}
  cos(x)*sin(y)
\end{euleroutput}
\begin{eulerprompt}
>plot3d(expr):
\end{eulerprompt}
\eulerimg{17}{images/Fransisca Renita_22305144012_Tugas 2-018.png}
\begin{eulercomment}
contoh 3:
\end{eulercomment}
\begin{eulerprompt}
>expr := "y^2-x^2"
\end{eulerprompt}
\begin{euleroutput}
  y^2-x^2
\end{euleroutput}
\begin{eulerprompt}
>aspect(1.5); plot3d(expr,-5,5,-5,5):
\end{eulerprompt}
\eulerimg{17}{images/Fransisca Renita_22305144012_Tugas 2-019.png}
\begin{eulercomment}
\begin{eulercomment}
\eulerheading{3. Menggambar Grafik Fungsi Dua Variabel yang}
\begin{eulercomment}
* Fungsinya Didefinisikan sebagai Fungsi Numerik

\end{eulercomment}
\eulersubheading{Fungsi Dua Variabel}
\begin{eulercomment}
Fungsi dua variabel adalah sebuah fungsi yang bernilai real dari dua
variabel real. Fungsi ini memadankan setiap pasangan terurut (x,y)
pada suatu himpunan D dari bidang dengan bilangan real tunggal f(x,y).
Dalam matematika, fungsi dua variabel atau lebih digunakan untuk
menggambarkan hubungan antara dua atau lebih variabel.

\end{eulercomment}
\eulersubheading{Fungsi Numerik}
\begin{eulercomment}
Fungsi numerik adalah suatu fungsi matematika yang menghasilkan nilai
numerik sebagai output-nya. Fungsi ini dapat dinyatakan dalam bentuk
persamaan matematika atau algoritma komputasi.

Contoh:

Fungsi\\
\end{eulercomment}
\begin{eulerformula}
\[
f(x,y) = 5x+y
\]
\end{eulerformula}
\begin{eulercomment}
Misal input nilai x=2 dan y=3, maka akan dihasilkan nilai z yaitu

\end{eulercomment}
\begin{eulerformula}
\[
z = f(x,y) = 5(2)+3 = 10+3 = 13
\]
\end{eulerformula}
\begin{eulercomment}
\end{eulercomment}
\eulersubheading{Gambar Grafik Fungsi}
\begin{eulercomment}
Fungsi satu baris numerik didefinisikan oleh ":=".

Langkah-langkah untuk memvisualisasikan grafik fungsi dua variabel
yang fungsi nya didefinisikan sebagai fungsi numerik dalam plot3d:

1. Buat fungsi numerik yang akan digunakan untuk memvisualisasikan
data.\\
function f(x,y):=ax+by\\
dimana a dan b adalah konstanta

2. Gunakan fungsi plot3d() untuk membuat grafik tiga dimensi dari
fungsi numerik.\\
plot3d("f"):

\end{eulercomment}
\eulersubheading{Contoh}
\begin{eulercomment}
Fungsi matematika f(x,y) dapat digambarkan dalam bentuk grafik tiga
dimensi menggunakan perintah plot3d. Berikut adalah contoh penggunaan
perintah plot3d untuk menggambarkan fungsi tersebut:

1. Fungsi Linear Dua Variabel

\end{eulercomment}
\begin{eulerformula}
\[
f(x,y)=5x+3y+1
\]
\end{eulerformula}
\begin{eulerprompt}
>function f(x,y):= 5*x+3*y+1
>plot3d("f"):
\end{eulerprompt}
\eulerimg{17}{images/Fransisca Renita_22305144012_Tugas 2-020.png}
\begin{eulercomment}
- Fungsi f(x,y) didefinisikan sebagai 5x+3y+1.\\
- Perintah "plot3d("f")" digunakan untuk memplot grafik 3D dari fungsi
f(x,y) menggunakan fungsi plot3d di EMT.\\
- Grafik yang dihasilkan akan menampilkan fungsi dalam tiga dimensi,
dengan sumbu x dan y mewakili variabel masukan dan sumbu z mewakili
nilai keluaran fungsi. Grafik akan menunjukkan bentuk fungsi dan
perubahannya seiring dengan perubahan variabel masukan.

\end{eulercomment}
\eulersubheading{}
\begin{eulercomment}
2. Fungsi Kuadrat Dua Variabel

\end{eulercomment}
\begin{eulerformula}
\[
f(x,y)=x^2+3y^2+21
\]
\end{eulerformula}
\begin{eulerprompt}
>function f(x,y):= x^2+(3*y)^2+27
>plot3d("f"):
\end{eulerprompt}
\eulerimg{17}{images/Fransisca Renita_22305144012_Tugas 2-021.png}
\begin{eulercomment}
- Perintah "function f(x,y):= x\textasciicircum{}2+(3*y)\textasciicircum{}2+27" berarti mendefinisikan
fungsi matematika f(x,y) sebagai x pangkat 2 ditambah 3 kali y pangkat
2 ditambah 27.\\
- Perintah "plot3d("f")" berarti membuat grafik tiga dimensi dari
fungsi f(x,y) yang telah didefinisikan sebelumnya.

\end{eulercomment}
\eulersubheading{}
\begin{eulercomment}
3. Fungsi Logaritma Dua Variabel

\end{eulercomment}
\begin{eulerformula}
\[
f(x,y)= \log(2xy)
\]
\end{eulerformula}
\begin{eulerprompt}
>function f(x,y):= log((2*x)*y)
>plot3d("f"):
\end{eulerprompt}
\eulerimg{17}{images/Fransisca Renita_22305144012_Tugas 2-022.png}
\begin{eulercomment}
- Input yang diberikan adalah fungsi matematika dua variabel, f(x,y),
yang didefinisikan sebagai logaritma hasil kali 2x dan y.\\
- Perintah "plot3d("f")" digunakan untuk memplot grafik fungsi f(x,y)
dalam ruang tiga dimensi.

\end{eulercomment}
\eulersubheading{}
\begin{eulercomment}
4. Fungsi Eksponen Dua Variabel

\end{eulercomment}
\begin{eulerformula}
\[
f(x,y)=x^{5y+10}
\]
\end{eulerformula}
\begin{eulerprompt}
>function f(x,y):= x^(5*y+10)
>plot3d("f"):
\end{eulerprompt}
\eulerimg{17}{images/Fransisca Renita_22305144012_Tugas 2-023.png}
\begin{eulercomment}
- Perintah `fungsi f(x,y):= x\textasciicircum{}(5y+10)` adalah fungsi matematika dua
variabel `x` dan `y` dan dengan rumus x\textasciicircum{}(5y+10)\\
- Perintah `plot3d("f")` digunakan untuk memplot fungsi dalam ruang
tiga dimensi. Plot yang dihasilkan akan menampilkan nilai fungsi
sebagai permukaan pada bidang x-y, dengan tinggi permukaan mewakili
nilai fungsi pada titik tersebut.

\end{eulercomment}
\eulersubheading{}
\begin{eulercomment}
5. Fungsi Trigonometri Dua Variabel

\end{eulercomment}
\begin{eulerformula}
\[
f(x,y)=cos(x)+sin(y)
\]
\end{eulerformula}
\begin{eulerprompt}
>function f(x,y):= cos(x)*sin(y)
>plot3d("f"):
\end{eulerprompt}
\eulerimg{17}{images/Fransisca Renita_22305144012_Tugas 2-024.png}
\begin{eulercomment}
- Perintah "function f(x,y):= cos(x)*sin(y)" adalah perintah untuk
mendefinisikan fungsi matematika f(x,y) yang menghasilkan nilai
cosinus dari x dikalikan dengan sinus dari y.\\
- Perintah "plot3d("f")" adalah perintah untuk membuat grafik tiga
dimensi dari fungsi f(x,y) yang telah didefinisikan sebelumnya.



\begin{eulercomment}
\eulerheading{4. Menggambar Grafik Fungsi Dua Variabel yang}
\begin{eulercomment}
* Fungsinya Didefinisikan sebagai Fungsi Simbolik

\end{eulercomment}
\eulersubheading{Fungsi Simbolik}
\begin{eulercomment}
Fungsi simbolik adalah fungsi yang dinyatakan dengan menggunakan
simbol-simbol matematika, seperti huruf dan operasi matematika,
daripada menggunakan angka konkret atau ekspresi numerik. Fungsi
simbolik sering digunakan untuk menggambarkan hubungan antara
variabel-variabel matematika dalam bentuk yang lebih umum dan abstrak.

Contoh fungsi simbolik yang umum adalah:

\end{eulercomment}
\begin{eulerformula}
\[
g(x,y) = 2x + y
\]
\end{eulerformula}
\begin{eulercomment}
Dalam contoh di atas, g(x) adalah fungsi simbolik yang mengaitkan
setiap nilai x dengan hasil dari ekspresi matematika 2x + 3. Fungsi
ini dapat digunakan untuk menghitung nilai fungsi untuk berbagai nilai
x.

\end{eulercomment}
\eulersubheading{Perbedaan Fungsi Numerik dan Fungsi Simbolik}
\begin{eulercomment}
1. Fungsi Numerik\\
Fungsi numerik dinyatakan dalam bentuk yang lebih konkret menggunakan
angka-angka nyata.

Contoh fungsi numerik adalah

\end{eulercomment}
\begin{eulerformula}
\[
g(x,y) = 2x + y + 3
\]
\end{eulerformula}
\begin{eulercomment}
dimana kita memberikan nilai numerik kepada "x dan y"\\
misalnya, x = 5 dan y = 2, maka hasilnya adalah angka konkret yaitu
g(5,2) = 15

2. Fungsi Simbolik\\
Fungsi simbolik dinyatakan menggunakan simbol-simbol matematika
seperti huruf (variabel) dan operasi matematika.

Contoh fungsi simbolik adalah

\end{eulercomment}
\begin{eulerformula}
\[
g(x,y) = 2x + y + 3
\]
\end{eulerformula}
\begin{eulercomment}
"g" adalah simbol fungsi\\
"x,y" adalah variabel,\\
2x + 3 adalah ekspresi matematika yang menggambarkan hubungan antara
"x,y" dan hasil fungsi.

\end{eulercomment}
\eulersubheading{Gambar Grafik Fungsi}
\begin{eulercomment}
Fungsi satu baris simbolik didefinisikan oleh "\&=".

Langkah-langkah untuk memvisualisasikan grafik fungsi dua variabel
yang fungsi nya didefinisikan sebagai fungsi simbolikdalam plot3d:

1. Buat fungsi simbolik yang akan digunakan untuk memvisualisasikan
data.\\
function g(x,y):= ax+by;\\
dimana a dan b adalah konstanta

2. Gunakan fungsi plot3d() untuk membuat grafik tiga dimensi dari
fungsi numerik.\\
plot3d("g"):

3. Menentukan rentang variabel\\
misal\\
plot3d("g(x,y)",-10,10,-5,5):\\
dengan batasan x dari -10 hingga 10 dan batasan y dari -5 hingga 5

\end{eulercomment}
\eulersubheading{Contoh}
\begin{eulercomment}
1. Fungsi Linear Dua Variabel

\end{eulercomment}
\begin{eulerformula}
\[
g(x,y)=x-2y+6
\]
\end{eulerformula}
\begin{eulercomment}
\end{eulercomment}
\begin{eulerprompt}
>function g(x,y)&= x-2*y+6;
>plot3d("g(x,y)"):
\end{eulerprompt}
\eulerimg{17}{images/Fransisca Renita_22305144012_Tugas 2-025.png}
\begin{eulercomment}
- Fungsi g(x,y) adalah fungsi matematika yang mengambil dua variabel,
x dan y, dan menghasilkan sebuah nilai berdasarkan rumus x - 2y + 6.\\
- Perintah "plot3d" digunakan untuk menghasilkan grafik tiga dimensi
dari fungsi tersebut.
\end{eulercomment}
\begin{eulerprompt}
>plot3d("g(x,y)",-1,2,0,2*pi):
\end{eulerprompt}
\eulerimg{17}{images/Fransisca Renita_22305144012_Tugas 2-026.png}
\begin{eulercomment}
- Perintah "plot3d("g(x,y)",-1,2,0,2*pi)" adalah perintah untuk
menggambar grafik fungsi tiga dimensi "g(x,y)" pada rentang x dari -1
hingga 2 dan rentang y dari 0 hingga 2pi.

\end{eulercomment}
\eulersubheading{}
\begin{eulercomment}
2. Fungsi Kuadrat Dua Variabel

\end{eulercomment}
\begin{eulerformula}
\[
g(x,y)=x^2+y^2+5
\]
\end{eulerformula}
\begin{eulerprompt}
>function g(x,y)&= x^2+y^2+5;
>plot3d("g(x,y)"):
\end{eulerprompt}
\eulerimg{17}{images/Fransisca Renita_22305144012_Tugas 2-027.png}
\begin{eulercomment}
- Fungsi g(x,y) adalah fungsi matematika yang mengambil dua variabel,
x dan y, dan menghasilkan sebuah nilai berdasarkan rumus x\textasciicircum{}2+y\textasciicircum{}2+5\\
- Perintah "plot3d" digunakan untuk menghasilkan grafik tiga dimensi
dari fungsi tersebut.
\end{eulercomment}
\begin{eulerprompt}
>plot3d("g(x,y)",-10,10,-1,5):
\end{eulerprompt}
\eulerimg{17}{images/Fransisca Renita_22305144012_Tugas 2-028.png}
\begin{eulercomment}
- Perintah "plot3d("g(x,y)",-10,10,-1,5)" adalah perintah untuk
menggambar grafik fungsi tiga dimensi g(x,y) pada rentang x dari -10
hingga 10 dan rentang y dari -1 hingga 5

\end{eulercomment}
\eulersubheading{}
\begin{eulercomment}
3. Fungsi Logaritma Dua Variabel

\end{eulercomment}
\begin{eulerformula}
\[
g(x,y) = \log(xy5)
\]
\end{eulerformula}
\begin{eulerprompt}
>function g(x,y)&= log(x*y*5);
>plot3d("g(x,y)"):
\end{eulerprompt}
\eulerimg{17}{images/Fransisca Renita_22305144012_Tugas 2-029.png}
\begin{eulercomment}
- Fungsi g(x,y) adalah fungsi matematika yang mengambil dua variabel,
x dan y, dan menghasilkan sebuah nilai berdasarkan rumus logaritma x
dikalikan y dikalikan 5\\
- Perintah "plot3d" digunakan untuk menghasilkan grafik tiga dimensi
dari fungsi tersebut.
\end{eulercomment}
\begin{eulerprompt}
>plot3d("g(x,y)",-1,1,-5,5):
\end{eulerprompt}
\eulerimg{17}{images/Fransisca Renita_22305144012_Tugas 2-030.png}
\begin{eulercomment}
- Perintah "plot3d("g(x,y)",-1,1,-5,5)" adalah perintah untuk
menggambar grafik fungsi tiga dimensi g(x,y) pada rentang x dari -1
hingga 1 dan rentang y dari -5 hingga 5
\end{eulercomment}
\begin{eulerprompt}
>plot3d("g(x,y)",-1,1,-5,5,zoom=4.5):
\end{eulerprompt}
\eulerimg{17}{images/Fransisca Renita_22305144012_Tugas 2-031.png}
\begin{eulercomment}
- plot3d: perintah untuk membuat grafik 3D.\\
- "g(x,y)": fungsi matematika yang akan digunakan untuk membuat
grafik.\\
- -1,1: rentang nilai variabel x yang akan digunakan dalam grafik.\\
- -5,5: rentang nilai variabel y yang akan digunakan dalam grafik.\\
- zoom=4.5: perintah untuk memperbesar tampilan grafik dengan faktor
4.5.

\end{eulercomment}
\eulersubheading{}
\begin{eulercomment}
4. Fungsi Eksponen Dua Variabel

\end{eulercomment}
\begin{eulerformula}
\[
g(x,y) = 2^{xy5}
\]
\end{eulerformula}
\begin{eulerprompt}
>function g(x,y)&= 2^(x*y*5);
>plot3d("g(x,y)"):
\end{eulerprompt}
\eulerimg{17}{images/Fransisca Renita_22305144012_Tugas 2-032.png}
\begin{eulercomment}
- Fungsi g(x,y) adalah fungsi matematika yang mengambil dua variabel,
x dan y, dan menghasilkan sebuah nilai berdasarkan rumus 2\textasciicircum{}(xy5)\\
- Perintah "plot3d" digunakan untuk menghasilkan grafik tiga dimensi
dari fungsi tersebut.
\end{eulercomment}
\begin{eulerprompt}
>plot3d("g(x,y)",-1,5,-1,3,frame=3,zoom=3):
\end{eulerprompt}
\eulerimg{17}{images/Fransisca Renita_22305144012_Tugas 2-033.png}
\begin{eulercomment}
- Peintah plot3d("g(x,y)",-1,5,-1,3,frame=3,zoom=3) adalah perintah
untuk membuat plot tiga dimensi dari fungsi `g(x,y)` dengan batas `x`
dari `-1` hingga `5` dan batas `y` dari `-1` hingga `3`.

- plot3d: perintah untuk membuat plot tiga dimensi.\\
- "g(x,y)"`: fungsi yang akan diplot.\\
- (-1,5): batas `x` dari `-1` hingga `5`.\\
- (-1,3): batas `y` dari `-1` hingga `3`.\\
- frame=3: menampilkan frame nomor 3.\\
- zoom=3: memperbesar tampilan plot sebanyak 3 kali.

\end{eulercomment}
\eulersubheading{}
\begin{eulercomment}
5. Fungsi Trigonometri Dua Variabel

\end{eulercomment}
\begin{eulerformula}
\[
g(x,y)=tan(x) - cot(y)
\]
\end{eulerformula}
\begin{eulerprompt}
>function g(x,y)&= tan(x)-cos(y);
>plot3d("g(x,y)"):
\end{eulerprompt}
\eulerimg{17}{images/Fransisca Renita_22305144012_Tugas 2-034.png}
\begin{eulercomment}
- Fungsi g(x,y) adalah fungsi matematika yang mengambil dua variabel,
x dan y, dan menghasilkan sebuah nilai berdasarkan rumus tan(x)-cos(y)\\
- Perintah "plot3d" digunakan untuk menghasilkan grafik tiga dimensi
dari fungsi tersebut.
\end{eulercomment}
\begin{eulerprompt}
>plot3d("g(x,y)",-1,3,0,2*pi,frame=1,zoom=3.5):
\end{eulerprompt}
\eulerimg{17}{images/Fransisca Renita_22305144012_Tugas 2-035.png}
\begin{eulercomment}
- Peintah plot3d("g(x,y)",-1,3,0,2*pi,frame=5,zoom=3) adalah perintah
untuk membuat plot tiga dimensi dari fungsi `g(x,y)` dengan batas `x`
dari `-1` hingga `3` dan batas `y` dari `0` hingga `2pi`.

- plot3d: perintah untuk membuat plot tiga dimensi.\\
- "g(x,y)"`: fungsi yang akan diplot.\\
- (-1,3): batas `x` dari `-1` hingga `3`.\\
- (0,2pi): batas `y` dari `0` hingga `2pi`.\\
- frame=1: menampilkan frame nomor 1.\\
- zoom=3.5: memperbesar tampilan plot sebanyak 3.5 kali.

\end{eulercomment}
\eulersubheading{}
\begin{eulercomment}
6. Fungsi Akar Kuadrat

\end{eulercomment}
\begin{eulerformula}
\[
P(x,y)= \sqrt{10x^2+2y^2}
\]
\end{eulerformula}
\begin{eulerprompt}
>function P(x,y) &= sqrt(10*x^2+2*y^2);
>plot3d("P(x,y)"):
\end{eulerprompt}
\eulerimg{17}{images/Fransisca Renita_22305144012_Tugas 2-036.png}
\begin{eulercomment}
- Fungsi P(x,y) adalah fungsi matematika yang mengambil dua variabel,
x dan y, dan menghasilkan sebuah nilai berdasarkan rumus akar kuadrat
dari 10x\textasciicircum{}2+2y\textasciicircum{}2\\
- Perintah "plot3d" digunakan untuk menghasilkan grafik tiga dimensi
dari fungsi tersebut.
\end{eulercomment}
\begin{eulerprompt}
>plot3d("P(x,y)",-2,2,0,3*pi,frame=5,zoom=2,scale=1):
\end{eulerprompt}
\eulerimg{17}{images/Fransisca Renita_22305144012_Tugas 2-037.png}
\begin{eulercomment}
- P(x,y): Merupakan fungsi yang akan digambarkan dalam grafik tiga
dimensi.\\
- (-2,2): Merupakan rentang nilai dari sumbu x yang akan digunakan
dalam grafik.\\
- (0,3pi): Merupakan rentang nilai dari sumbu y yang akan digunakan
dalam grafik. Nilai pi dikalikan dengan 3 agar rentang nilai y
mencakup tiga putaran lingkaran penuh.\\
- frame=5: Menentukan nomor bingkai (frame) yang akan digunakan dalam
animasi grafik.\\
- zoom=2: Menentukan faktor pembesaran grafik. Dengan memperbesar
tampilan, kita dapat melihat detail yang lebih kecil pada plot.\\
- scale=1: Menentukan skala grafik. Dengan mengatur skala, kita dapat
mengubah jarak antara titik-titik pada sumbu tersebut.
\end{eulercomment}
\begin{eulercomment}
1. aspect(1.5) mengatur aspek rasio pada grafik 3D.\\
2. plot3d(expr,-5,5,-5,5) adalah fungsi matematika yang digunakan
untuk membuat grafik 3D.\\
3. -5,5 mengatur rentang sumbu x yang akan ditampilkan pada grafik.\\
4. -5,5 mengatur rentang sumbu y yang akan ditampilkan pada grafik.
\end{eulercomment}
\eulerheading{Menggambar Data $x$, $y$, $z$}
\begin{eulercomment}
* pada ruang Tiga Dimensi (3D)

Definisi

\end{eulercomment}
\begin{eulerttcomment}
  Menggambar data pada ruang tiga dimensi (3D) adalah proses
\end{eulerttcomment}
\begin{eulercomment}
visualisasi data yang mengubah informasi dalam tiga dimensi, yaitu
panjang, lebar, dan tinggi, menjadi representasi visual yang dapat
dipahami dan dianalisis.

Tujuan:

\end{eulercomment}
\begin{eulerttcomment}
  Tujuan dari menggambar data 3D adalah untuk membantu pemahaman dan
\end{eulerttcomment}
\begin{eulercomment}
interpretasi data yang lebih baik, terutama ketika data tersebut
memiliki komponen yang tidak dapat direpresentasikan dengan baik dalam
dua dimensi.

Sama seperti plot2d, plot3d menerima data. Untuk objek 3D, Anda perlu
menyediakan matriks nilai x-, y- dan z, atau tiga fungsi atau ekspresi
fx(x,y), fy(x,y), fz(x,y).

\end{eulercomment}
\begin{eulerformula}
\[
\gamma(t,s) = (x(t,s),y(t,s),z(t,s))
\]
\end{eulerformula}
\begin{eulercomment}
Karena x,y,z adalah matriks, kita asumsikan bahwa (t,s) melalui sebuah
kotak persegi. Hasilnya, Anda dapat memplot gambar persegi panjang di
ruang angkasa.

Kita dapat menggunakan bahasa matriks Euler untuk menghasilkan
koordinat secara efektif.

Dalam contoh berikut, kami menggunakan vektor nilai t dan vektor kolom
nilai s untuk membuat parameter permukaan bola. Dalam gambar kita
dapat menandai daerah, dalam kasus kita daerah kutub.

\end{eulercomment}
\eulersubheading{Contoh 1}
\begin{eulerprompt}
>t=-1:0.1:1; s=(-1:0.1:1)'; plot3d(t,s,t*s,grid=10):
\end{eulerprompt}
\eulerimg{17}{images/Fransisca Renita_22305144012_Tugas 2-038.png}
\begin{eulercomment}
Baris pertama kode "t=-1:0.1:1" membuat vektor baris t yang berisi
nilai dari -1 hingga 1 dengan interval 0.1. Baris kedua
"s=(-1:0.1:1)'" membuat vektor kolom s yang berisi nilai dari -1
hingga 1 dengan interval 0.1. Operator transpose ' digunakan untuk
mengubah vektor baris t menjadi vektor kolom.\\
Baris ketiga "plot3d(t,s,ts,grid=10)" membuat plot tiga dimensi dari
fungsi f(x,y) = xy pada domain [-1,1] x [-1,1]. Plot dibuat
menggunakan fungsi plot3d, yang mengambil tiga argumen: koordinat x,
y, dan z dari titik-titik yang akan diplot. Dalam hal ini, koordinat x
diberikan oleh vektor t, koordinat y diberikan oleh vektor s, dan
koordinat z diberikan oleh hasil perkalian t dan s, yaitu ts.
Parameter grid diatur menjadi 10, yang menunjukkan jumlah garis grid
yang akan ditampilkan pada plot.

\end{eulercomment}
\eulersubheading{Contoh 2}
\begin{eulercomment}
Tentu saja, titik cloud juga dimungkinkan. Untuk memplot data titik
dalam ruang, kita membutuhkan tiga vektor untuk koordinat titik-titik
tersebut.

Gayanya sama seperti di plot2d dengan points=true;
\end{eulercomment}
\begin{eulerprompt}
>n=500;...
>plot3d(normal(1,n),normal(1,n),normal(1,n),points=true,style="."):
\end{eulerprompt}
\eulerimg{17}{images/Fransisca Renita_22305144012_Tugas 2-039.png}
\begin{eulercomment}
Kode "n=500;
plot3d(normal(1,n),normal(1,n),normal(1,n),points=true,style=".")"
digunakan untuk membuat plot tiga dimensi dari tiga vektor normal yang
dihasilkan secara acak dengan menggunakan fungsi "normal()" pada Euler
Math Toolbox (EMT). Parameter "n=500" menunjukkan bahwa setiap vektor
normal memiliki 500 elemen. Parameter "points=true" digunakan untuk
menampilkan titik-titik pada plot, sedangkan parameter "style='.'"
digunakan untuk mengatur gaya titik pada plot menjadi titik bulat.

\end{eulercomment}
\eulersubheading{Contoh 3}
\begin{eulercomment}
\end{eulercomment}
\begin{eulerttcomment}
 Dengan lebih banyak usaha, kami dapat menghasilkan banyak permukaan.
\end{eulerttcomment}
\begin{eulercomment}
\end{eulercomment}
\begin{eulerttcomment}
 Dalam contoh berikut, kita membuat tampilan bayangan dari bola yang
\end{eulerttcomment}
\begin{eulercomment}
terdistorsi. Koordinat biasa untuk bola adalah

\end{eulercomment}
\begin{eulerformula}
\[
\gamma(t,s) = (\cos(t)\cos(s),\sin(t)\sin(s),\cos(s))
\]
\end{eulerformula}
\begin{eulercomment}
dengan

\end{eulercomment}
\begin{eulerformula}
\[
0 \le t \le 2\pi, \quad \frac{-\pi}{2} \le s \le \frac{\pi}{2}.
\]
\end{eulerformula}
\begin{eulercomment}
Kami mendistorsi ini dengan sebuah faktor

\end{eulercomment}
\begin{eulerformula}
\[
d(t,s) = \frac{\cos(4t)+\cos(8s)}{4}
\]
\end{eulerformula}
\begin{eulerprompt}
>t=linspace(0,2pi,320); s=linspace(-pi/2,pi/2,160)';...
>d=1+0.2*(cos(4*t)+cos(8*s));...
>plot3d(cos(t)*cos(s)*d,sin(t)*cos(s)*d,sin(s)*d,hue=1,...
>light=[1,0,1],frame=0,zoom=5):
\end{eulerprompt}
\eulerimg{17}{images/Fransisca Renita_22305144012_Tugas 2-040.png}
\begin{eulercomment}
Kode ini terdiri dari beberapa baris. Baris pertama
"t=linspace(0,2pi,320)" membuat vektor t yang berisi 320 nilai yang
sama terdistribusi secara merata antara 0 dan 2p. Baris kedua
"s=linspace(-pi/2,pi/2,160)'" membuat vektor s yang berisi 160 nilai
yang sama terdistribusi secara merata antara -p/2 dan p/2. Operator
transpose ' digunakan untuk mengubah vektor baris s menjadi vektor
kolom.

Baris ketiga "d=1+0.2*(cos(4t)+cos(8s))" membuat vektor d yang berisi
nilai dari 1 + 0.2 * (cos(4t) + cos(8s)). Baris keempat
"plot3d(cos(t)*cos(s)*d,sin(t)*cos(s)*d,sin(s)*d,hue=1,light=[1,0,1],frame=0,zoom=5)"
membuat plot tiga dimensi dari fungsi f(x,y) = 2x\textasciicircum{}2 + y\textasciicircum{}3. Plot dibuat
menggunakan fungsi plot3d, yang mengambil empat argumen: koordinat x,
y, dan z dari titik-titik yang akan diplot, serta beberapa parameter
lainnya. Dalam hal ini, koordinat x diberikan oleh ekspresi
cos(t)*cos(s)*d, koordinat y diberikan oleh ekspresi sin(t)*cos(s)*d,
dan koordinat z diberikan oleh ekspresi sin(s)*d. Parameter "hue=1"
digunakan untuk mengatur warna pada plot berdasarkan nilai fungsinya.
Parameter "light=[1,0,1]" digunakan untuk mengatur pencahayaan pada
plot. Parameter "frame=0" digunakan untuk menghilangkan frame pada
plot. Parameter "zoom=5" digunakan untuk mengatur level zoom pada
plot.
\end{eulercomment}
\eulerheading{Grafik Tiga Dimensi yang}
\begin{eulercomment}
* Bersifat Interaktif dan animasi grafik 3D
\end{eulercomment}
\begin{eulercomment}
Membuat gambar grafik tiga dimensi (3D) yang bersifat interaktif dan
animasi grafik 3D adalah proses menciptakan visualisasi tiga dimensi
yang memungkinkan pengguna berinteraksi dengan objek-objek 3D.
Interaktivitas dalam gambar 3D memungkinkan pengguna untuk melakukan
tindakan seperti mengubah sudut pandang, memindahkan objek, atau
berinteraksi dengan elemen-elemen dalam adegan 3D. Animasi grafik 3D
dapat mencakup pergerakan, tetapi juga dapat berarti perubahan dalam
tampilan atau atribut objek tanpa pergerakan fisik yang mencolok.

CONTOH GAMBAR
\end{eulercomment}
\begin{eulerprompt}
>function testplot () := plot3d("x^2+y^3"); ...
>rotate("testplot"); testplot(): 
\end{eulerprompt}
\eulerimg{17}{images/Fransisca Renita_22305144012_Tugas 2-041.png}
\begin{eulerprompt}
>function testplot () := plot3d("x^2+y",distance=3,zoom=1,angle=pi/2,height=0); ...
>rotate("testplot"); testplot(): 
\end{eulerprompt}
\eulerimg{17}{images/Fransisca Renita_22305144012_Tugas 2-042.png}
\begin{eulercomment}
Hilangkan command angle untuk bisa merotasikan grafik,dan height = 0
untuk membuat posisi sejajar dengan mata jadi tidak mempengaruhi
pergerakan hanya berbeda sudut pandang saja
\end{eulercomment}
\begin{eulerprompt}
>plot3d("exp(-x^2+y^2)",>user, ...
>  title="Turn with the vector keys (press return to finish)"):
\end{eulerprompt}
\eulerimg{17}{images/Fransisca Renita_22305144012_Tugas 2-043.png}
\begin{eulerprompt}
>plot3d("exp(x^2+y^2)",>user, ...
>title="Coba gerakan)")
\end{eulerprompt}
\begin{eulercomment}
Interaksi pengguna dimungkinkan dengan parameter. Pengguna dapat
menekan tombol berikut.\\
1. kiri, kanan, atas, bawah: memutar sudut pandang\\
2. +,-: memperbesar atau memperkecil\\
3. a: menghasilkan anaglyph (lihat di bawah)\\
4. l: beralih memutar sumber cahaya (lihat di bawah)\\
5. spasi: disetel ulang ke default\\
6. enter: akhiri interaksi
\end{eulercomment}
\begin{eulerprompt}
>plot3d("exp(-(x^2+y^2)/5)",r=10,n=80,fscale=4,scale=1.2,frame=3,>user):
\end{eulerprompt}
\eulerimg{17}{images/Fransisca Renita_22305144012_Tugas 2-044.png}
\begin{eulercomment}
Parameter "r=10" menunjukkan jari-jari bola yang digunakan untuk
membuat plot tiga dimensi. Dalam hal ini, jari-jari bola yang
digunakan adalah 10.\\
Parameter "n=80" menunjukkan jumlah titik yang digunakan untuk membuat
plot. Semakin besar nilai n, semakin banyak titik yang digunakan untuk
membuat plot, sehingga plot akan menjadi lebih halus dan akurat.\\
Parameter "fscale=4" menunjukkan faktor skala pada sumbu z. Dalam hal
ini, faktor skala pada sumbu z adalah 4.\\
Parameter "scale=1.2" menunjukkan faktor skala pada plot. Semakin
besar nilai scale, semakin besar ukuran plot yang dihasilkan.\\
Parameter "frame=3" menunjukkan jenis frame yang digunakan pada plot.
Dalam hal ini, jenis frame yang digunakan adalah frame kotak dengan
sumbu x, y, dan z yang ditampilkan.
\end{eulercomment}
\begin{eulerprompt}
>plot3d("x^2+y",distance=3,zoom=1,angle=pi/2,height=0):
\end{eulerprompt}
\eulerimg{17}{images/Fransisca Renita_22305144012_Tugas 2-045.png}
\begin{eulercomment}
Tampilan dapat diubah dengan berbagai cara.

- distance: jarak pandang ke plot.\\
- zoom: nilai zoom.\\
- angle: sudut terhadap sumbu y negatif dalam radian.\\
- height: ketinggian tampilan dalam radian.
\end{eulercomment}
\begin{eulerprompt}
>plot3d("x^4+y^2",a=0,b=1,c=-1,d=1, angle=-20?, height=20?, ...
>  center=[0.4,0,0], zoom=5):
\end{eulerprompt}
\begin{euleroutput}
  Closing bracket missing in function call!
  Error in:
  plot3d("x^4+y^2",a=0,b=1,c=-1,d=1, angle=-20?, height=20?,   c ...
                                              ^
\end{euleroutput}
\begin{eulercomment}
Plot selalu terlihat berada di tengah kubus plot. Anda dapat
memindahkan bagian tengah dengan parameter center.

Parameter center digunakan untuk memindahkan pusat plot ke lokasi
tertentu dalam ruang. Dalam hal ini, pusat plot diatur ke titik (0.4,
0, 0) dalam ruang tiga dimensi. Parameter center berguna ketika kita
ingin mengubah sudut pandang plot atau ketika kita ingin menyelaraskan
plot dengan objek lain dalam scene. Dengan menentukan pusat plot, kita
dapat mengontrol posisi kamera dan arah tampilan plot.

Ada beberapa parameter untuk menskalakan fungsi atau mengubah tampilan
grafik.

fscale: menskalakan ke nilai fungsi (defaultnya adalah \textless{}fscale).\\
scale: angka atau vektor 1x2 untuk diskalakan ke arah x dan y.\\
frame: jenis bingkai (default 1).
\end{eulercomment}
\begin{eulerprompt}
>function testplot () := plot3d("5*exp(-x^2-y^2)",r=2,<fscale,<scale,distance=13,height=50?, ...
>center=[0,0,-2],frame=3); ...
>rotate("testplot"); testplot():
\end{eulerprompt}
\begin{euleroutput}
  Closing bracket missing in function call!
  testplot:
      useglobal; return plot3d("5*exp(-x^2-y^2)",r=2,<fscale,<scale ...
  Try "trace errors" to inspect local variables after errors.
  rotate:
      f$(args());
\end{euleroutput}
\begin{eulerprompt}
>plot3d("x^2+1",a=-1,b=1,rotate=true,grid=5):
\end{eulerprompt}
\eulerimg{17}{images/Fransisca Renita_22305144012_Tugas 2-046.png}
\begin{eulercomment}
Penjelasan:\\
Secara umum, parameter "a" dan "b" digunakan untuk menentukan rentang
nilai variabel independen dalam suatu fungsi. Dalam kasus ini, "a=-1"
dan "b=1" menunjukkan bahwa fungsi tersebut akan diplot pada interval
[-1, 1]. Parameter "rotate=true" menunjukkan bahwa grafik akan diputar
untuk memberikan tampilan bentuk tiga dimensi yang lebih baik.
Parameter "grid=5" menunjukkan bahwa grid dengan jarak 5 unit akan
ditampilkan pada grafik.

Parameter memutar memutar fungsi dalam x di sekitar sumbu x.

- rotate=1: Menggunakan sumbu x\\
- rotate=2: Menggunakan sumbu z
\end{eulercomment}
\begin{eulerprompt}
>plot3d("x^2+1",a=-1,b=1,rotate=2,grid=5):
\end{eulerprompt}
\eulerimg{17}{images/Fransisca Renita_22305144012_Tugas 2-047.png}
\begin{eulerprompt}
>function testplot () := plot3d("sqrt(25-x^2)",a=0,b=5,rotate=1); ...
>rotate("testplot"); testplot():
\end{eulerprompt}
\eulerimg{17}{images/Fransisca Renita_22305144012_Tugas 2-048.png}
\begin{eulerprompt}
>function testplot () := plot3d("x^4+y^2",a=0,b=1,c=-1,d=1,height=20?, ...
>center=[0.4,0,0],zoom=5); ...
>rotate("testplot"); testplot():
\end{eulerprompt}
\begin{euleroutput}
  Closing bracket missing in function call!
  testplot:
      useglobal; return plot3d("x^4+y^2",a=0,b=1,c=-1,d=1,height=20 ...
  Try "trace errors" to inspect local variables after errors.
  rotate:
      f$(args());
\end{euleroutput}
\begin{eulerprompt}
>function testplot () := plot3d("1/(x^2+y^2+1)",r=5,>polar, ...
>fscale=2,>hue,n=100,zoom=4,>contour,color=red); ...
>rotate("testplot"); testplot():
\end{eulerprompt}
\eulerimg{17}{images/Fransisca Renita_22305144012_Tugas 2-049.png}
\begin{eulercomment}
Parameter "r=5" menunjukkan jari-jari bola yang digunakan untuk
membuat plot tiga dimensi. Dalam hal ini, jari-jari bola yang
digunakan adalah 5.\\
Parameter "\textgreater{}polar" menunjukkan bahwa plot yang dibuat adalah plot
polar tiga dimensi. Plot polar adalah plot yang dibuat dengan
menggunakan koordinat polar, yaitu koordinat yang terdiri dari jarak
dan sudut.\\
Parameter "fscale=2" menunjukkan faktor skala pada sumbu z. Dalam hal
ini, faktor skala pada sumbu z adalah 2.\\
Parameter "\textgreater{}hue" menunjukkan bahwa warna pada plot akan diatur
berdasarkan nilai fungsinya. Semakin tinggi nilai fungsinya, semakin
terang warnanya.\\
Parameter "n=100" menunjukkan jumlah titik yang digunakan untuk
membuat plot. Semakin besar nilai n, semakin banyak titik yang
digunakan untuk membuat plot, sehingga plot akan menjadi lebih halus
dan akurat.\\
Parameter "zoom=4" menunjukkan level zoom pada plot.\\
Parameter "\textgreater{}contour" menunjukkan bahwa garis kontur akan ditampilkan
pada plot.\\
Parameter "color=blue" menunjukkan warna garis kontur pada plot. Dalam
hal ini, warna yang digunakan adalah biru.

Untuk plotnya, Euler menambahkan garis grid. Sebaliknya dimungkinkan
untuk menggunakan garis level dan satu warna atau warna spektral.
Euler dapat menggambar ketinggian fungsi pada sebuah plot dengan
bayangan. Di semua plot 3D, Euler dapat menghasilkan anaglyph
merah/cyan.

-hue: Mengaktifkan bayangan cahaya, bukan kabel.\\
-contour: Membuat plot garis kontur otomatis pada plot.\\
-level=... (atau level): Vektor nilai garis kontur.
\end{eulercomment}
\begin{eulerprompt}
>function testplot () := plot3d("x^2-y^2",0,5,0,5,level=-1:0.1:1,color=blue); ...
>rotate("testplot"); testplot():
\end{eulerprompt}
\eulerimg{17}{images/Fransisca Renita_22305144012_Tugas 2-050.png}
\begin{eulercomment}
Parameter "level=-1:0.1:1" menunjukkan rentang nilai fungsinya yang
akan ditampilkan pada plot. Dalam hal ini, rentang nilai fungsinya
adalah dari -1 hingga 1 dengan interval 0.1.
\end{eulercomment}
\begin{eulerprompt}
>function testplot () := plot3d("x^2+y^4",>cp,cpcolor=green,cpdelta=0.2); ...
>rotate("testplot"); testplot():
\end{eulerprompt}
\eulerimg{17}{images/Fransisca Renita_22305144012_Tugas 2-051.png}
\begin{eulercomment}
Parameter "\textgreater{}cp" menunjukkan bahwa titik kontrol akan ditambahkan pada
plot. Titik kontrol digunakan untuk menentukan bentuk dan posisi plot
tiga dimensi.\\
Parameter "cpcolor=green" menunjukkan warna titik kontrol yang akan
digunakan. Dalam hal ini, warna yang digunakan adalah hijau.\\
Parameter "cpdelta=0.2" menunjukkan jarak antara titik kontrol.
Semakin kecil nilai cpdelta, semakin banyak titik kontrol yang akan
ditambahkan pada plot.
\end{eulercomment}
\begin{eulerprompt}
>plot3d("-x^2-y^2", ...
>hue=true,light=[0,1,1],amb=0,user=true, ...
> title="Press l and cursor keys (return to exit)"):
\end{eulerprompt}
\eulerimg{17}{images/Fransisca Renita_22305144012_Tugas 2-052.png}
\begin{eulercomment}
parameter "hue=true" menunjukkan bahwa warna pada plot akan diatur
berdasarkan nilai fungsinya. Semakin tinggi nilai fungsinya, semakin
terang warnanya.\\
Parameter "light=light=[0,1,1] menunjukkan intensitas cahaya pada
plot. Nilai light=[0,1,1] menunjukkan bahwa cahaya datang dari arah
positif y dan z.\\
Parameter "amb=0" menunjukkan intensitas cahaya ambient pada plot.
Nilai 0 menunjukkan bahwa tidak ada cahaya ambient yang digunakan.
\end{eulercomment}
\begin{eulerprompt}
>function testplot () := plot3d("-x^2-y^2",color=rgb(0.2,0.2,0),hue=true,frame=false, ...
> zoom=3,contourcolor=red,level=-2:0.1:1,dl=0.01); ...
>rotate("testplot"); testplot():
\end{eulerprompt}
\eulerimg{17}{images/Fransisca Renita_22305144012_Tugas 2-053.png}
\begin{eulercomment}
Parameter "frame=false" digunakan untuk menghilangkan frame pada plot
tiga dimensi. Parameter "color=rgb(0.2,0.2,0)" menunjukkan warna dasar
plot. Dalam hal ini, warna yang digunakan adalah hitam dengan nilai
RGB (0.2, 0.2, 0). Parameter "dl=0.01" menunjukkan jarak antara
titik-titik pada plot. Semakin kecil nilai dl, semakin banyak titik
yang digunakan untuk membuat plot, sehingga plot akan menjadi lebih
halus dan akurat. Namun, semakin kecil nilai dl, semakin lama waktu
yang dibutuhkan untuk membuat plot.
\end{eulercomment}
\begin{eulerprompt}
>function testplot () := plot3d("x^2+y^3",>contour,>spectral); ...
>rotate("testplot"); testplot():
\end{eulerprompt}
\eulerimg{17}{images/Fransisca Renita_22305144012_Tugas 2-054.png}
\begin{eulerprompt}
>function testplot () := plot3d("x^2+y^3", >transparent, grid=10, wirecolor=red); ...
>rotate("testplot"); testplot():
\end{eulerprompt}
\eulerimg{17}{images/Fransisca Renita_22305144012_Tugas 2-055.png}
\eulerheading{Fungsi Parametrik 3D}
\begin{eulercomment}
Fungsi parametrik merupakan jenis fungsi matematika yang menggambarkan
hubungan antara dua atau lebih variabel, dimana masing-masing
koordinat (x, y, z...) dinyatakan sebagai fungsi lain dari beberapa
parameter. Fungsi parametrik dapat digunakan untuk menggambar kurva,
lintasan, atau hubungan antara berbagai variabel yang bergantung pada
parameter-parameter tertentu.

Sebagai contoh :
\end{eulercomment}
\begin{eulerprompt}
>plot3d("cos(x)*cos(y)","sin(x)*cos(y)","sin(y)", a=0,b=2*pi,c=pi/2,d=-pi/2,...
>>hue,color=blue,light=[0,1,3],<frame,...
>n=90,grid=[20,50],wirecolor=black,zoom=5):
\end{eulerprompt}
\eulerimg{17}{images/Fransisca Renita_22305144012_Tugas 2-056.png}
\begin{eulerprompt}
>plot3d("cos(x)*cos(y)","sin(x)*cos(y)","cos(x)", a=0,b=2*pi,c=pi/2,d=-pi/2,...
>>hue,color=blue,light=[0,1,3],<frame,...
>n=90,grid=[20,50],wirecolor=black,zoom=5):
\end{eulerprompt}
\eulerimg{17}{images/Fransisca Renita_22305144012_Tugas 2-057.png}
\begin{eulerprompt}
>plot3d("cos(x)^3*sin(y)","sin(x)^2*sin(y)","cos(x)^2", a=0,b=2*pi,c=pi/2,d=-pi/2,...
>>hue,color=blue,light=[0,1,5],<frame,...
>n=90,grid=[20,50],wirecolor=black,zoom=5):
\end{eulerprompt}
\eulerimg{17}{images/Fransisca Renita_22305144012_Tugas 2-058.png}
\eulerheading{8 Menggambar Fungsi Implisit Implisit}
\begin{eulercomment}
Fungsi implisit (implicit function) adalah fungsi yang memuat lebih
dari satu variabel, berjenis variabel bebas dan variabel terikat yang
berada dalam satu ruas sehingga tidak bisa dipisahkan pada ruas yang
berbeda.

\end{eulercomment}
\begin{eulerformula}
\[
F(x,y,z)=0
\]
\end{eulerformula}
\begin{eulercomment}
(1 persamaan dan 3 variabel), terdiri dari 2 variabel bebas dan 1
terikat

Misalnya,\\
\end{eulercomment}
\begin{eulerformula}
\[
F(x, y, z) = x^2 + y^2 + z^2 = 1
\]
\end{eulerformula}
\begin{eulercomment}
adalah persamaan implisit yang menggambarkan bola dengan jari-jari 1
dan pusat di (0,0,0).

\end{eulercomment}
\begin{eulerprompt}
>plot3d("x^2+y^3+z*y-1", r=5, implicit=3):
\end{eulerprompt}
\eulerimg{17}{images/Fransisca Renita_22305144012_Tugas 2-059.png}
\begin{eulerprompt}
>c=1; d=1;
>plot3d("((x^2+y^2-c^2)^2+(z^2-1)^2)*((y^2+z^2-c^2)^2+(x^2-1)^2)*((z^2+x^2-c^2)^2+(y^2-1)^2)-d", r=2, <frame,>implicit,>user):
\end{eulerprompt}
\eulerimg{17}{images/Fransisca Renita_22305144012_Tugas 2-060.png}
\begin{eulerprompt}
>plot3d("x^2+y^2+4*x*z+z^3",>implicit, r=2, zoom=2.5):
\end{eulerprompt}
\eulerimg{17}{images/Fransisca Renita_22305144012_Tugas 2-061.png}
\begin{eulercomment}
Selain plot kontur yang sudah di jelaskan sebelumnya, pada EMT juga
ada plot umplisit dalam tiga dimensi. Euler menghasilkan potongan
melalui objek. Fitur plot3d termasuk plot implisit. Plot-plot ini
menunjukkan himpunan nol dari sebuah fungsi dalam tiga variabel.

Solusi dari\\
\end{eulercomment}
\begin{eulerformula}
\[
f(x,y,z) = 0
\]
\end{eulerformula}
\begin{eulercomment}
dapat divisualisasikan dalam potongan yang sejajar dengan bidang x-y,
bidang x-z, dan bidang y-z.

- implisit = 1: potong sejajar dengan bidang-y-z\\
- implicit = 2: memotong sejajar dengan bidang x-z\\
- implicit=4: memotong sejajar dengan bidang x-y

Ambil contoh dari persamaan latex pada fungsi implisit tadi dan
tambahkan nilai-nilai ini, sehingga kita dapat memplot persamaan ini\\
\end{eulercomment}
\begin{eulerformula}
\[
M = {(x,y,z) :{ x^2+y^3+zy=1}}
\]
\end{eulerformula}
\begin{eulerprompt}
>plot3d("x^2+y^3+z*y", r=1, implicit=2):
\end{eulerprompt}
\eulerimg{17}{images/Fransisca Renita_22305144012_Tugas 2-062.png}
\begin{eulercomment}
Contoh fungsi implisit yang lainnya
\end{eulercomment}
\begin{eulerprompt}
>plot3d("x^3+y^3+z*y-1",r=7,implicit=4):
\end{eulerprompt}
\eulerimg{17}{images/Fransisca Renita_22305144012_Tugas 2-063.png}
\begin{eulerprompt}
>plot3d("2*x^2 + 3*y^2 + z^2 - 25",r=8,implicit=2):
\end{eulerprompt}
\eulerimg{17}{images/Fransisca Renita_22305144012_Tugas 2-064.png}
\begin{eulerprompt}
>plot3d("x^5 + 5*y^3 + 3*z^2 - 5*x - 7*y - 5*z + 10",r=5,implicit=2):
\end{eulerprompt}
\eulerimg{17}{images/Fransisca Renita_22305144012_Tugas 2-065.png}
\begin{eulerprompt}
>plot3d("x^3+y^5+5*x*z+z^3",>implicit,r=3,zoom=2):
\end{eulerprompt}
\eulerimg{17}{images/Fransisca Renita_22305144012_Tugas 2-066.png}
\begin{eulerprompt}
>plot3d("x^2+y^2+4*x*z+z^3-5",>implicit,r=2,zoom=2.5):
\end{eulerprompt}
\eulerimg{17}{images/Fransisca Renita_22305144012_Tugas 2-067.png}
\eulerheading{Fungsi Implisit Menggunakan Povray}
\begin{eulercomment}
Povray dapat memplot himpunan di mana f(x,y,z)=0, seperti parameter
implisit di plot3d. Namun, hasilnya terlihat jauh lebih baik.

Sintaks untuk fungsi-fungsi tersebut sedikit berbeda. Anda tidak dapat
menggunakan output dari ekspresi Maxima atau Euler.

\end{eulercomment}
\begin{eulerformula}
\[
((x^2+y^2-c^2)^2+(z^2-1)^2)*((y^2+z^2-c^2)^2+(x^2-1)^2)*((z^2+x^2-c^2)^2+(y^2-1)^2)=d
\]
\end{eulerformula}
\begin{eulerprompt}
>load povray;
>defaultpovray="C:\(\backslash\)Program Files\(\backslash\)POV-Ray\(\backslash\)v3.7\(\backslash\)bin\(\backslash\)pvengine.exe"
\end{eulerprompt}
\begin{euleroutput}
  C:\(\backslash\)Program Files\(\backslash\)POV-Ray\(\backslash\)v3.7\(\backslash\)bin\(\backslash\)pvengine.exe
\end{euleroutput}
\begin{eulerprompt}
>povstart(angle=70°,height=50°, zoom=4);
>writeln(povsurface("pow(x,2)*y-pow(y,3)-pow(z,2)",povlook(blue)));
>writeAxes();
>povend();
\end{eulerprompt}
\begin{euleroutput}
  Command was not allowed!
  exec:
      return _exec(program,param,dir,print,hidden,wait);
  povray:
      exec(program,params,defaulthome);
  Try "trace errors" to inspect local variables after errors.
  povend:
      povray(file,w,h,aspect,exit); 
\end{euleroutput}
\end{eulernotebook}
\end{document}
